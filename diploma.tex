% В этом шаблоне используется класс spbau-diploma. Его можно найти и, если требуется, 
% поправить в файле spbau-diploma.cls
\documentclass{spbau-diploma}
\newcommand{\todo}{\textcolor{red}}

\begin{document}
% Год, город, название университета и факультета предопределены,
% но можно и поменять.
% Если англоязычная титульная страница не нужна, то ее можно просто удалить.
\filltitle{ru}{
    chair              = {Кафедра математических и информационных технологий},
    title              = {Решение задачи восстановления клональных деревьев репертуара антител с помощью данных парного иммуносеквенирования},
    % Здесь указывается тип работы. Возможные значения:
    %   coursework - Курсовая работа
    %   diploma - Диплом специалиста
    %   master - Диплом магистра
    %   bachelor - Диплом бакалавра
    type               = {master},
    position           = {студента},
    group              = 605,
    author             = {Черниговская Мария Александровна},
    supervisorPosition = {к.\,ф.-м.\,н., постдок \\Калифорнийского университета в Сан-Диего\\},
    supervisor         = {Сафонова Я.\,Ю.},
    reviewerPosition   = {??\\},
    reviewer           = {Шугай М.\,А.},
    chairHeadPosition  = {д.\,ф.-м.\,н., профессор},
    chairHead          = {Омельченко А.\,В.},
    % university = {САНКТ-ПЕТЕРБУРГСКИЙ АКАДЕМИЧЕСКИЙ УНИВЕРСИТЕТ},
    % faculty = {Центр высшего образования},
    % city = {Санкт-Петербург},
    % year             = {2013}
}
\filltitle{en}{
    chair              = {Department of Mathematics and Information Technology},
    title              = {Clonal trees reconstraction of antobody repertoires using paired single cell sequencing data},
    author             = {Maria Chernigovskaya},
    supervisorPosition = {Ph.D., postdoc at UCSD\\},
    supervisor         = {Yana Safonova},
    reviewerPosition   = {???\\},
    reviewer           = {Mikhail Shugay},
    chairHeadPosition  = {professor},
    chairHead          = {Alexander Omelchenko},
}
\maketitle
\tableofcontents

\section*{Реферат}

В данной работе описывается алгоритм  pairedAntEvolo, который позволяет анализировать данные парного иммуносеквенирования, полученные методом секвенированмя одиночных клеток.
Алгоритм предназначен для разбиения репертуара B-клеток на клональные линии и восстановления эволюционных деревьев внутри каждой линии.
 \\

\textbf{Ключевые слова:} парное иммуносеквенирование, секвенирование одиночных клеток, клональные линии, клональные деревья, B-клетки, репертуар антител, созревание лимфоцитов, адаптивный иммунный ответ

% -------------------------------------------------------------------------------------------

\section{Введение}

% Какие-то вводные слова
Развитие иммунной системы обусловило возможность существования todo{сложно организованных многоклеточных организмов}.
Она распознает множество разнообразных возбудителей, от вирусов до паразитических червей, и отличает их от биомолекул клеток.
Конечной целью работы иммунной системы является ликвидация чужеродного агента, которым может оказаться болезнетворный микроорганизм, инородное тело, ядовитое вещество или переродившаяся клетка самого организма.
Различают врожденную и адаптивную иммуную систему.
Задача врожденной иммунной системы давать быстрый иммунный ответ на ограниченное число патогенов, но она может справиться не со всем.
Например, есть бактерии, которые умеют быстро мутировать. Для того, чтобы справляться с ними, иммунной системе необходимо адаптироваться -- так у организмов начиная с рыб появилась \textit{адаптивная имунная система}, которая не только
умеет обучаться и отвечать практически на что угодно, но и запоминает результат иммунного ответа.


% Мы будем смотреть на В-клетки
Важнейшим элементом адаптивной иммунной системы являются В-клетки.
В-клетки производят специальные белки ---  антитела, которые циркулируют в крови и в лимфе.
Антитела распознают антигены -- поверхностные белки вирусов и бактерий -- связываются с ними и участвуют в их нейтрализации.


% Устройство антитела + VDJ-рекомбинация
Молекула антитела состоит их двух пар различных белков, которые называются тяжелой и легкой цепями   \todo{(тут не уверен)}
Вариабельные части тяжелой и легкой цепи образуют сайт связывания с антигеном и определяют специфичность данного антитела, то есть определяют, с каким именно антигеном будет связываться конкретное антитело.
Таким образом, для того, чтобы имунная система могла бороться с огромным количеством потенциальных антигенов, В-клетки должны уметь синтезировать большое количество различных антител.
Так как количество возможных антигенов много больше числа генов, которые можно закодировать в ДНК, существуют специальные процессы, которые повышают разнообразие антител: VDJ-рекомбинация, спаривание тяжелой и легкой цепи, соматический гипермутагенез и клональная селекция.



\begin{figure}[h!]
    \centering
    \includegraphics[width=.9\textwidth]{figures/10x_vdj_recombination.png}
    \caption{\textbf{ПЕРЕРИСОВАТЬ} Схематичное описание VDJ-рекомбинации.
    В геноме присутствуют три группы генов: V, D и J гены, которые похожи внутри каждой группы.
    После рекомбинации тяжелая цепь содержит в себе по одному конкретному гену V, D и J, легкая цепь --- только V и J.
    Сперва рекомбинирует тяжелая цепь: на первом этапе рекомбинации вырезается случайное количество D-сегментов с конца и случайное количество J-сегментов с начала.
    На втором этапе вырезается случайное количество конечных V-сегментов и все D-сегменты, кроме последнего.
    На стыках V-D и D-J происходят случайные мутации и вставки.
    Лишние V- и J-гены затем вырезаются в процессе транскрипции.
    В итоге, оставшиеся V-, D и J- гены вместе с константной частью образуют ген тяжелой цепи иммуноглобулина.
    Получившаяся в ходе рекомбинации тяжелая цепь тестируется на продуктивность.
    Если цепь не работает, то рекомбинирует вторая аллель.
    Если обе цепи не работают, то В-клетка погибает.
    Затем, аналогично тяжелой цепи, рекомбинирует легкая цепь.
    Если рекомбинация прошла успешно, то белки тяжелой и легкой цепь соединяются вместе и образуют антитело.
    сегменты.}
    \label{10x_vdj_recombination}
\end{figure}

% Откуда получаются деревья
VDJ-рекомбинация уникальным образом меняет геном гомопоэтической стволовой клетки и превращает ее в В-клетку.
После того, как В-клетка успешно связалась с антигеном, она перемещается в герминальный центр и начинает делиться.
В получившихся клетках-клонах специальный фермент вносит соматические гипермутации в вариабельную часть антигена, для того, чтобы улучшить связываемость антитела с антигеном.
Клетки с нефункциональными рецепторами и клетки подозрительные на аутоиммунные отсеиваются, полезные клетки стимулируются к дальшейшему делению и мутациям.
В результате такого эволюционного процесса образуются клональные семейства В-клеток.


% Rep-seq vs single-cell
Для того, чтобы изучать имунную систему, существует несколько технологий секвенирования, которые позволяют восстанавливать вариабельную часть генов иммуноглобулина.
Наиболее распространенное семейство технологий называется Rep-seq (Repertoire sequencing, ~\cite{pmid22043864}), которое позволяет с очень высокой точностью определить частоты антител встречающихся в образце.
(Что-то написать про саму технологию(??).
В целом, это обычное секвенирование кДНК с правильно подобранными праймерами, которые садятся на консервативный кусок в начало V-гена).
Недостатком этих технологий является то, что в процессе секвенирования теряется важная информация о парности цепей, то есть какие именно тяжелые и легкие цепи образовывали вместе антитело.
Не так давно появилась технология, которая позволяет секвенировать РНК антигенов с помощью одиночных клеток и сохраняет информацию о парности цепей.
Для этого к РНК пришиваются молекулярные и клеточные баркоды.
К сожалению, на текущий момент не существует общепринятого и стабильного протокола, но пока коммерческая компания 10х genomics являются лидерами по качеству парных данных.

% Цель работы
Целью данной работы является анализ реальных данных парного иммуносеквенирования и разработка метода восстановления клональных деревьев с помощью данных этого вида.



% -------------------------------------------------------------------------------------------

\subsection{Мотивация}

% Репертуар имеет структуру! Это не просто множество, а дерево
В большинстве исследований \todo{(найти статью)} репертуар рассматривается как множество антител, однако клональные деревья позволяют использовать дополнительную информацию о его структуре.
При активном имунном ответе происходит активное деление полезных В-клеток, и скорее всего важные антитела будут содержаться в самых больших деревьях.
Также построение клональных деревьев по данным временных серий позволяет изучать динамику имунного ответа \todo{(найти статью,~\cite{stern2014b} не совсем то)}.
Для того, чтобы установить, в какой день наблюдался самый сильный иммунный ответ, можно следить за изменением размера самых больших деревьев.
Также можно изучать, какие именно мутации оказались самыми полезными и привели к наибольшему распространению клона.


% Зачем строить клональные деревья
А еще лучше смотреть на парный репертуар, чтобы не строить мифические деревья, потому что есть allelic inclusion.
И мы не можем собрать антитело, пока не знаем обе цепи (и даже если знаем, то тоже сложно).


% -------------------------------------------------------------------------------------------

\subsection{Постановка задачи}

Биологическая и математическая

% Проанализировать данные

% Разработать алгоритм

% -------------------------------------------------------------------------------------------

\subsection{Существующие решения}

% Для одной цепи
Все умеют строить по одной цепи.


% Для двух цепей (bracer)
По двум тоже умеют, но очень плохо.

% -------------------------------------------------------------------------------------------

\section{Анализ реальных данных}

% Описание технологии
На текущий момент компания 10x genomics~\cite{pmid28091601} является лидером по производству парных данных.
Их технология секвенирования одиночных иммунных клеток (рис.\ref{10x_pipeline}) позволяет с высокой точностью обработать от 100 до 10000 клеток в одном семпле \todo{образце?}.
Для того, чтобы запомнить информацию о том, какие цепи экспрессирует конкретная клетка, используют клеточные и молекулярные баркоды (рис.\ref{10x_barcodes}) .
Клеточные баркоды позволяют отметить последовательности, которые относятся к одной клетке.
 Молекулярные баркоды нужны для того, чтобы восстановить последовательности после амплификации.
 Помеченные двумя баркодами последовательности секвенируются с помощью парных ридов (150 нуклеотидов), которые затем собираются в контиги с помощью разработанного в компании инструмента CellRanger VDJ.
Затем происходит аннотация --- контиги выравниваются на базу известных V, D, J сегментов (гермлайн), и последовательности с плохим выравниванием отфильтровываются.
Для оставшихся последовательностей определяется лучшее выравнивание на V и J сегменты.
В конечном итоге получается набор последовательностей цепей иммуноглобулинов с клеточным баркодом и предполагаемыми V и J сегментами, из которые были выбраны в ходе VDJ-рекомбинации.

\begin{figure}[h!]
    \centering
    \includegraphics[width=.9\textwidth]{figures/10x_pipeline.png}
    \caption{Описание технологии 10x genomics VDJ: На вход поступает клеточная суспензия определенной плотности.
    Клетки вместе с реагентами реакции проходят через микрофлюидный (микрогидродинамический) канал, смешиваются с гелевыми шариками и лизисным буфером из другого канала и формируют капли, изолированные друг от друга маслом.
    На поверхности гелевых шариков находятся баркодированные олигонуклеотиды.
    Внутри капли за секунды происходит лизис клетки, и содержимое клетки, в том числе молекулы матричной РНК, оказывается внутри капли.
    Молекулы мРНК гибридизуются с олигонуклеотидами на поверхности шариков, и затем внутри капли происходит обратная транскрипция молекул мРНК в молекулы кДНК.
    Затем капли <<лопаются>>, полученная библиотека кДНК амплифицируется с помощью ПЦР и отправляется на секвенирование}
    \label{10x_pipeline}
\end{figure}

\begin{figure}[h!]
    \centering
    \includegraphics[width=.9\textwidth]{figures/10x_barcodes.png}
    \caption{Структура библиотеки: левый рид покрывает клеточный (16 нуклеотидов) и молекулярный (10 нуклеотидов) баркод.
    Оба рида являются стандартными парными ридами для технологии Illumina и используются для того, чтобы покрыть вставку.}

    \label{10x_barcodes}
\end{figure}

% Сколько цепей может экспрессировать В-клетка
Известно, что при созревании В-лимфоцитов происходит аллельное исключение, то есть успешная VDJ-рекомбинация гена, который кодирует тяжёлую цепь иммуноглобулина на одной хромосоме, блокирует  рекомбинацию в гомологичной хромосоме.
Однако существует исследование~\cite{barreto2000frequency}, в котором было показано, что у здоровых мышей в $1$ случае из $10000$ может происходить одновременная экспрессия обеих тяжелых цепей в IgM+ В-клетках селезенки.
Авторы статьи предложили несколько моделей, которые могли бы объяснить данное явление, но пока эти гипотезы не подтверждены.
У человека неизветны случаи, в которых В-клетки экспрессируют несолько тяжелых цепей, но есть исследование, в котором было показано, что $30\%$ клеток гибридомы (гибридная клеточная линия, которая получена в результате слияния B-лимфоцитов, полученных из селезёнки иммунизированного животного и опухолевых клеток миеломы) имели дополнительные тяжелые и легкие цепи.
Причем  $1.
\%$ клеток имели дополнительную легкую цепь, а $2.
\%$ --- дополнительную тяжелую и легкую цепь.
На текущий момент проведено множество исследований (~\cite{pelanda2014dual}, ~\cite{casellas2007igkappa}, ~\cite{liu2005receptor}, ~\cite{fraser2015immunoglobulin}), которые подтверждают, что В-клетки могут экспрессировать две легкие цепи как одного, так и разных типов.
Большинство таких В-клеток экспрессируют две легкие каппа-цепи \todo{вроде бы не было сказано, кто такие каппа и лямбда?}, но редко могут экспрессировать две лямбда-цепи и цепи разных типов.
Обычно одно экспрессируемое антитело получается аутореактивное, а второе нормальное, которое позволяет В-клетке обмануть негативную селекцию, созреть, активироваться и дифференцироваться.
Считается, что В-клетки  с несолькими легкими цепями встречаются в ???? (малом, меньше $1\%$) случаях.


% Что мы ожидаем от данных
Таким образом, мы ожидаем увидеть в парных данных, что большинство В-клеток экспрессируют одну легкую и одну тяжелую цепь или (редко) одну тяжелую и две легких цепи.
При секвенировании мы можем увидеть не все цепи, так как они могли не проэкспрессироваться на момент эксперимента.
Поэтому мы также ожидаем увидеть в данных клетки, которые имеют только легкую или только тяжелую цепь.
Клетки, у которых просеквенирован нестандартный набор цепей, например несколько тяжелых и больше двух легких цепей, скорее всего являются артефактом секвенирования и получились в результате коллизии, при которой больше одной клетки попадали в каплю с молекулярными баркодами.
10х genomics утверждают~\cite{10x_manual}, что вероятность коллизии зависит от концентрации клеток в семпле, и варьируется от $0.
\%$ до $7.
\%$.

% Описание датасетов
Для того, чтобы изучить специфику парных данных 10x, были проанализированы публичные В-клеточные датасеты~\cite{10x_datasets}: 
\begin{enumerate}
    \item CD19 --- B-клетки с маркером CD19, которые были выделены из мононуклеарных клеток периферической крови здоровых доноров.
    Так как этот вид клеток экспрессирует небольшое количество РНК, библиотека прошла целевое обогащение;
    \item GM12878 --- В-лимфобластоидная клеточная линия с высоким уровнем транскрипции иммуноглобулинов;
    \item NSCLC --- целевое обогащение иммуноглобулинов B-клеток, которые были получены во время свежего хирургического удаления немелкоклеточного рака легкого.
\end{enumerate}

Помимо стандартных типов цепей иммуноглобулинов --- тяжелая, легкая каппа и легкая лямбда, аннотируют последовательности еще одним типом --- мультицепь.
Мультицепь --- это последовательность, которая выравнивается на гены разных типов, например V ген выравнивается на каппу-цепь, а J ген на лямбду.
На практике оказалось, что в мультицепи попадают последовательности, которые были неправильно проаннотированы.
Например, последовательность TGACGGCAGTGAACGC\-1\_contig\_4 из датасета CD19 (подтверждена $83$ молекулярными баркодами) в аннотации 10х имеет IGLV3-21 и IGHJ6 гены, хотя IgBLAST проаннотировал этот контиг генами IGLV3-21*02 и IGLJ1*01 с хорошим качеством выравнивания.
Также в аннотациях 10х встречаются последовательности, которые содержат одновременно T и В-клеточные гены, но на самом деле они содержат лишь какие-то подпоследовательности генов иммуноглобулинов (рис.\ref{IgBLAST_multi}).
Мы предполагаем, что в парные данные также попадает РНК, которая транскрибируется из иммунного локуса, и 10х не отфильтровывает такие контиги, а пытается проаннотировать.


\begin{figure}[h!]
  \centering
  \begin{subfigure}{\linewidth}
    \centering
    \includegraphics[width=1\textwidth]{figures/IgBLAST_bcr.png}
    \caption{Аннотация контига IgBlAST с выравниванием на базу В-клеточных генов иммуноглобулинов}
  \end{subfigure}

  \begin{subfigure}{\linewidth}
    \centering
    \includegraphics[width=1\textwidth]{figures/IgBLAST_tcr.png}
    \caption{Аннотация контига IgBlAST с выравниванием на базу Т-клеточных генов иммуноглобулинов}
  \end{subfigure}  
  \caption{Пример последовательности, которая проаннотирована 10х как мультицепь с Т-клеточным геном TRAJ22  и В-клеточными геном IGKV2D-28 (контиг GAACATCTCGGCGGTT\-1\_contig\_4 из датасета CD19, подтвержден $65$ баркодами).
  Из аннотации IgBLAST видно, что последовательность плохо выравнивается на обе базы генов, а также в ней не содержатся гены, которые были указаны 10х.}

  \label{IgBLAST_multi}
\end{figure}  

Во всех датасетах мы дополнительно отфильтровали контиги, которые подтверждены небольшим количеством молекулярных баркодов (меньше $10$), потому что такие контиги с большой вероятностью являются шумом, например, это могут быть контиги внеклеточной РНК.

Для того, чтобы проверить точность аннотации данных с помощью 10х, мы сравнили ее с IgBLAST, который считается золотым стандартом аннотации иммуных последовательностей.
Результат сравнения приведен в таблице~\ref{10x_vs_IgBLAST}.
Из таблицы видно, что ДОПИСАТЬ.

\begin{table}[h!]
\centering
\begin{tabular}{|l|l|l|l|}
\hline
                    & \textbf{CD19} & \textbf{GM12878} & \textbf{NSCLC} \\ \hline
\# клеток в датасете                           &      &         &       \\ \hline
\# всех цепей в датасете                       &      &         &       \\ \hline
\% цепей отфильтрованных IgBLAST               &      &         &       \\ \hline
\% мультицепей                                 &      &         &       \\ \hline
\% мультицепей, которые проаннотировал IgBLAST &      &         &       \\ \hline
\% цепей с совпавшими аннотациями V генов      &      &         &       \\ \hline
\% цепей с совпавшими аннотациями J генов      &      &         &       \\ \hline
\% цепей с совпавшими аннотациями V и J генов  &      &         &       \\ \hline
\end{tabular}
\caption{Сравнение 10х аннотатора и IgBLAST}
\label{10x_vs_IgBLAST}
\end{table}

% Статы по количеству цепей в клетке
Из таблицы~\ref{stats_nchains} видно, что существенный процент клеток имеет $3$ цепи, что противоречит нашим ожиданиям.
Для того, чтобы подтвердить, что это действительно разные цепи, а не артефакты технологии секвенирования и ПЦР, мы построили гистограмму расстояний между двумя цепями одного типа, которые находятся в одной клетке.
Из графиков~\ref{chains_hist} видно, что цепи делятся на два кластера разного размера, но при этом цепи в большем кластере сильно различаются.
Таким образом, в парных данных 10х большее число клеток с тремя цепями, чем ожидалось из предыдущих исследований.

\begin{table}[h!]
\centering
\begin{tabular}{|l|l|l|l|l|l|l|l|}
\hline
        & \textbf{1 цепь} & \textbf{2 цепи} & \textbf{3 цепи} & \textbf{4 цепи} & \textbf{5 цепей} & \textbf{6 цепей} & \textbf{\textgreater{}6 цепей} \\ \hline
CD19    & 22.67\%         & 63.26\%         & 10.53\%         & 2.7\%           & 0.69\%           & 0.14\%           & 0.01\%                         \\ \hline
GM12878 & 17.51\%         & 56.80\%         & 21.66\%         & 4.03\%          & 0.00\%           & 0.00\%           & 0.00\%                         \\ \hline
NSCLC   & 51.26\%         & 46.55\%         & 2.02\%          & 0.17\%          & 0.00\%           & 0.00\%           & 0.00\%                         \\ \hline
\end{tabular}
\caption{Процентное соотношение клеток по количеству экспрессируемых цепей}
\label{stats_nchains}
\end{table}

 
 \begin{figure}[h!]
    \centering
    \includegraphics[width=.9\textwidth]{figures/HKK_hist.png}
    \caption{ПЕРЕРИСОВАТЬ НОРМАЛЬНО И ДЛЯ ВСЕХ (конкретно этот случай для 1 тяжелой и 2 каппы). Распределение расстояния между кратными цепями в клетках, которые содержат $3$ цепи.  В качестве расстояния между двумя цепями было взято расстояние редактирования со следующими штрафами: $0$ за совпадение, $-1$ за несовпадение, $-0.5$ за открытие гэпа и $-0.1$ за его продолжение.}
    \label{chains_hist}
\end{figure}

   

% Статы по типам клеток






% -------------------------------------------------------------------------------------------

\section{Симулятор парных данных}

Так как реальные данные странноваты, мы будем моделировать (непонятно, как ведут себя несколько легких цепей с точки зрения дерева.
Отбор идет только по одной или обеим?).
Частоты цепей возьмем из статьи~\cite{dekosky2015depth}.



% -------------------------------------------------------------------------------------------

\section{Описание алгоритма}

В процессе осмысления



% У заключения нет номера главы
\section*{Заключение}

\bibliographystyle{ugost2008ls}
\bibliography{diploma.bib}
\end{document}
